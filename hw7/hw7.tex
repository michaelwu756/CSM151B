\documentclass[12pt]{article}
\usepackage{amsmath}
\begin{document}
\title{Computer Science M151B, Homework 7}
\date{May 21st, 2018}
\author{Michael Wu\\UID: 404751542}
\maketitle

\section*{Problem 1}

For the transfer using 4-word blocks, each transfer will take 1 cycle to send the address to memory, 24 cycles to access memory for the first
set of words, 1 cycle to send the data from memory, and 2 idle clock cycles. So each 4-word block takes 28 cycles to read. This is 140 ns for each
4-word block. Thus the sustained bandwidth of the read is
\[4\,\frac{\text{words}}{\text{block}}\times 4\,\frac{\text{bytes}}{\text{word}}\times \frac{1 \text{ block}}{1.4 \times 10^{-7} \text{ s}}
=1.1423 \times 10^8\,\frac{\text{bytes}}{\text{s}}\]
The latency to transfer 256 words is then
\[\frac{256 \text{ words}}{4\,\frac{\text{words}}{\text{block}}}\times 140\,\frac{\text{ns}}{\text{block}}=8960 \text{ ns}\]
and the number of bus transactions per second is
\[\frac{1 \text{ transaction}}{1.4 \times 10^{-7} \text{ s}}=7.143 \times 10^{6}\,\frac{\text{transactions}}{\text{s}}\]

For the transfer using 16-word blocks, each transfer will take 1 cycle to send the address to memory, 24 cycles to access memory for the first
set of words, 1 cycle for each 4-word block to send the data from memory, and 2 idle clock cycles for each 4-word block. So each 16-word block
takes 37 cycles to read. This is 185 ns for each
16-word block. Thus the sustained bandwidth of the read is
\[16\,\frac{\text{words}}{\text{block}}\times 4\,\frac{\text{bytes}}{\text{word}}\times \frac{1 \text{ block}}{1.85 \times 10^{-7} \text{ s}}
=3.4595 \times 10^8\,\frac{\text{bytes}}{\text{s}}\]
The latency to transfer 256 words is then
\[\frac{256 \text{ words}}{16\,\frac{\text{words}}{\text{block}}}\times 185\,\frac{\text{ns}}{\text{block}}=2960 \text{ ns}\]
and the number of bus transactions per second is
\[\frac{1 \text{ transaction}}{1.85 \times 10^{-7} \text{ s}}=5.405 \times 10^{6}\,\frac{\text{transactions}}{\text{s}}\]

\section*{Problem 2}

In DMA-based I/O, an interrupt must still be given to the CPU to indicate that the device controller has finished an I/O request and put some data in memory.
The difference between interrupt-driven I/O and DMA-based I/O is that the interrupt-driven I/O will send an interrupt for every piece of data. The data in
interrupt-driven I/O always passes through the CPU prior to going to memory, whereas in DMA-based I/O the device controller can directly modify memory.
Thus DMA-based I/O introduces complexity into a computer's design because it may cause cache coherency problems. If the device controller modifies data in memory
that has changed somewhere higher in the cache, the CPU may accidentally use the wrong data which would lead to incorrect behaviour.

Interrupt-driven I/O would be preferred for simpler devices where the performance penalty is small compared to DMA-based I/O. If a lot of data is not being transferred
to memory at a time, then the advantage of DMA-based I/O being able to pass data quickly to memory is diminished. For example, if there is a sensor that generates one byte
every second and sends it to the processor, then interrupt-driven I/O would be preferred. This is because the device does not use a lot of CPU time, so the performance improvement
of DMA-based I/O would not be significant. If DMA-based I/O were used, this would add unecessary complexity.

\section*{Problem 3}

\paragraph{a)}

\paragraph{b)}

\section*{Problem 4}

\paragraph{a)}

\paragraph{b)}

\paragraph{c)}

\paragraph{d)}

\section*{Problem 5}

\section*{Problem 6}

\end{document}