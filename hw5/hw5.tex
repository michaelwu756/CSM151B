\documentclass[12pt]{article}
\usepackage{graphicx}
\begin{document}
\title{Computer Science M151B, Homework 5}
\date{May 7th, 2018}
\author{Michael Wu\\UID: 404751542}
\maketitle

\section*{Problem 1}

Yes, the \texttt{PCSrc} signal could be computed in the \texttt{EX} stage, as the zero of the ALU output and the \texttt{Branch} signal are both
available in \texttt{EX}. An advantage of making this change would be to reduce the branch penalty from three cycles to two cycles, since we can
update the program counter one cycle earlier. However, since this adds an \texttt{and} gate to the output of the ALU, this could potentially
lead to a higher cycle time. The book shows a better implementation where the branch computed in the \texttt{ID} stage, reducing the branch penalty
to one cycle, which is why this change is not used.

\section*{Problem 2}

\paragraph{a)}

The value of \texttt{PCSrc} is 0 when this instruction is in the \texttt{MEM} stage. This is because there is no branching going on, so the program counter
should update normally.

\paragraph{b)}

The value of \texttt{PCSrc} is 1 when this instruction is in the \texttt{MEM} stage. This is because this branch instruction checks if register \texttt{\$5}
is equal to itself, which must be true. So the branch will be taken and the program counter will be updated to the address given in the branch instruction.

\section*{Problem 3}

This series of instructions will execute over 9 cycles. On the fifth cycle, the first \texttt{add} instruction is in the \texttt{WB} stage, so register \texttt{\$2} is
written. Additionally, the third \texttt{add} instruction is in the \texttt{ID} stage, so the registers \texttt{\$6} and \texttt{\$1} are being read.

\section*{Problem 4}

\begin{figure}[!ht]
        \hspace*{-4cm}
        \begin{center}
                \includegraphics[width=5in]{problem4.png}
        \end{center}
        \hspace*{-4cm}
\end{figure}

The modified hazard detection unit is shown above. The goal of this modification is to always stall for three cycles after every branch.
I have added the \texttt{Branch} control signal to the \texttt{MEM/WB} registers, so that a signal is available to determine if a branch instruction
is in the \texttt{WB} stage. I have also added the \texttt{MEM/WB.PCSrc} signal to aid in fetching a new instruction if the branch is taken. The hazard detection
unit stalls by injecting 0 into the \texttt{ID/EX} registers. It also sets \texttt{IF/IDWrite} and \texttt{PCWrite} to 0 to prevent a new instruction
from being loaded.

The logic for the outputs of the hazard detection unit is shown below.
\begin{verbatim}
if (ID/EX.Branch or EX/MEM.Branch or MEM/WB.Branch)
  then ID/EXZero = 1 else ID/EXZero = 0

if (MEM/WB.PCSrc != 1 and
  (ID/EX.Branch or EX/MEM.Branch or MEM/WB.Branch))
    then IF/IDWrite = 0 else IF/IDWrite = 1

if (MEM.PCSrc != 1 and MEM/WB.PCSrc != 1 and
  (ID/EX.Branch or EX/MEM.Branch or MEM/WB.Branch))
    then PCWrite = 0 else PCWrite = 1
\end{verbatim}
Note that I set \texttt{PCWrite = 1} when \texttt{MEM.PCSrc = 1} to allow for the branch to update the program counter if it is taken. I then
set \texttt{PCWrite = 1} and \texttt{IF/IDWrite = 1} when \texttt{MEM/WB.PCSrc = 1} to allow for the instruction at the branch target address to be read, and the
program counter to be updated in the next cycle.

\section*{Problem 5}

\paragraph{a)}

\paragraph{b)}

\section*{Problem 6}

\section*{Problem 7}

\section*{Problem 8}

The opcode must contain at least 8 bits, and each register field must contain at least 6 bits. So that leaves at most 12 bits for the immediate field.
Thus the maximum range of the immediate value is from \(-2048\) to \(2047\) in decimal.

\section*{Problem 9}

\end{document}