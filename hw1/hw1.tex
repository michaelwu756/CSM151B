\documentclass[12pt]{article}
\usepackage{amsmath}
\usepackage{amssymb}
\begin{document}
\title{Computer Science M151B, Homework 1}
\date{April 9th, 2018}
\author{Michael Wu\\UID: 404751542}
\maketitle

\section*{Problem 1}
\section*{Problem 2}

\begin{verbatim}
addi $9,  $5,  -12
add  $10, $5,  $8
lw   $11, 3880220($10)
sub  $11, $11, $13
sw   $11, 3880220($9)
\end{verbatim}

\section*{Problem 3}

\begin{verbatim}
andi $8,  $14, 0x0003FC00
sll  $8,  $8,  14
andi $22, $22, 0x00FFFFFF
or   $22, $22, $8
\end{verbatim}

\section*{Problem 4}

\begin{verbatim}
slt  $20, $0, $17 # $0 < $17, so $20=1
bne  $20, $0, 1   # $20 != $0, so increase program counter by 1
beq  $0, $0, 1    # skipped
addi $20, $20, 2  # $20 = $20 + 2, so $20=3
\end{verbatim}

The final value of register \texttt{\$20} is \(3\). The flow of execution is such that the first instruction stores \(1\) in register
\texttt{\$20}, which then causes the next instruction to branch. This skips the third instruction, and finally the last instruction adds
\(2\) to register \texttt{\$20} and stores it, resulting in the value of \(3\).

\section*{Problem 5}
\section*{Problem 6}
\section*{Problem 7}
\section*{Problem 8}
\section*{Problem 9}

\end{document}
