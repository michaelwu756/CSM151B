\documentclass[12pt]{article}
\usepackage{amsmath}
\usepackage{amssymb}
\begin{document}
\title{Computer Science M151B, Homework 1}
\date{April 9th, 2018}
\author{Michael Wu\\UID: 404751542}
\maketitle

\section*{Problem 1}
\section*{Problem 2}

\begin{verbatim}
addi $9,  $5,  -12
add  $10, $5,  $8
sll  $9,  $9,  2
sll  $10, $10, 2
lui  $11, 0x351C
slr  $11, $11, 16
lui  $11, 0x3B
add  $9,  $9,  $11
add  $10, $10, $11
lw   $12, 0($10)
sub  $12, $12, $13
sw   $12, 0($9)
\end{verbatim}

\section*{Problem 3}

\begin{verbatim}
addi $8,  $0,  0xFF
sll  $8,  $8,  10
and  $9,  $14, $8
sll  $9,  $9,  14
sll  $8,  $8,  14
and  $22, $22, $8
or   $22, $22, $9
\end{verbatim}

\section*{Problem 4}

\begin{verbatim}
slt  $20, $0, $17 # $0 < $17, so $20=1
bne  $20, $0, 1   # $20 != $0, so increase program counter by 1
beq  $0, $0, 1    # skipped
addi $20, $20, 2  # $20 = $20 + 2, so $20=3
\end{verbatim}

The final value of register \texttt{\$20} is \(3\). The flow of execution is such that the first instruction stores \(1\) in register
\texttt{\$20}, which then causes the next instruction to branch. This skips the third instruction, and finally the last instruction adds
\(2\) to register \texttt{\$20} and stores it, resulting in the value of \(3\).

\section*{Problem 5}

The machine code in binary is
\begin{verbatim}
1001 0110 1100 1110 0000 0000 0010 1101
\end{verbatim}

\section*{Problem 6}

\begin{center}
        \begin{tabular}{|c||c|c|c|c|c|c|c|}
                \hline
                Address & op & rs & rt & rd & shamt & func & inst \\
                \hline\hline
                0000 0001 1100 0100 & 001000 & 00000 & 01111 & \multicolumn{3}{|c|}{0000 0000 0010 0001} & addi\\
                \hline
                0000 0001 1100 1000 & 000000 & 00000 & 00101 & 00100 & 00010 & 000000 & sll\\
                \hline
                0000 0001 1100 1100 & 000000 & 01011 & 00100 & 00100 & 00000 & 100000& add\\
                \hline
                0000 0001 1101 0000 & 100011 & 00100 & 00100 & \multicolumn{3}{|c|}{0000 0000 0000 0000} & lw\\
                \hline
                0000 0001 1101 0100 & 000100 & 00100 & 01111 & \multicolumn{3}{|c|}{0000 0000 0000 0111} & beq\\
                \hline
                0000 0001 1101 1000 & 001000 & 00101 & 00101 & \multicolumn{3}{|c|}{0000 0000 0000 0001} & addi\\
                \hline
                0000 0001 1101 1100 & 000000 & 00111 & 00100 & 00011 & 00000 & 101010 & slt\\
                \hline
                0000 0001 1110 0000 & 000100 & 00011 & 00000 & \multicolumn{3}{|c|}{0000 0000 0000 0010} & beq\\
                \hline
                0000 0001 1110 0100 & 000000 & 01100 & 00100 & 01100 & 00000 & 100000 & add\\
                \hline
                0000 0001 1110 1000 & 000010 & \multicolumn{5}{|c|}{00 0000 0000 0000 0000 0111 0010} & j\\
                \hline
                0000 0001 1110 1100 & 000000 & 01100 & 00111 & 01100 & 00000 & 100010 & sub\\
                \hline
                0000 0001 1111 0000 & 000010 & \multicolumn{5}{|c|}{00 0000 0000 0000 0000 0111 0010} & j\\
                \hline
        \end{tabular}
\end{center}

\section*{Problem 7}



\section*{Problem 8}
\section*{Problem 9}

\end{document}
